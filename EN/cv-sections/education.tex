%----------------------------------------------------------------------------------------
%	SECTION TITLE
%----------------------------------------------------------------------------------------

\cvsection{Formation}

%----------------------------------------------------------------------------------------
%	SECTION CONTENT
%----------------------------------------------------------------------------------------

\begin{cventries}

%------------------------------------------------

  \cventrysep{2018 - 2019}{%
    \vspace{-.0cm}
\cventry
{Master 2 MOSIG -- Mention Bien\\(Master of Science in Informatics at Grenoble)\\Troisième année de magistère informatique} % Degree
{Université Grenoble Alpes, UFR IM$^2$AG, Saint Martin d'Hères, France} % Institution
{} % Location
{} % Date(s)
{
  \begin{cvdescription}
    \cvinternship{Stage de Master 2 et Magistère, Laboratoire TIMA (Grenoble)}%
    {Vers l'analyse automatisée de tolérance aux fautes pour automates
      programmables}%
    {Étude sur la modélisation de scénarios de test nominaux et de
      scénarios de fautes, et génération automatique de tests
      exécutables.\\
      Technologies utilisées :
      \vspace{.4cm}
      \begin{cvitems}
      \item OCaml
      \item Microsoft Z3 SAT/SMT solver
      \item Réseaux de Petri
      \item GRAFCET
      \end{cvitems}
      \vspace{.4cm}
    }%
    {Février 2019 - Août 2019}
  \end{cvdescription}
}
}

\vspace{.5cm}

\cventrysep[4.2cm]{2017 - 2018}{%
  \vspace{-0.55cm}
\cventry
{Master 1 Informatique -- Mention Bien\\Deuxième année de magistère informatique} % Degree
{Université Grenoble Alpes, UFR IM$^2$AG, Saint Martin d'Hères, France} % Institution
{} % Location
{} % Date(s)
{
  \begin{cvdescription}
    \cvinternship{Stage de Magistère, Laboratoire TIMA (Grenoble)}%
    {Solutions to implement safety/security online monitoring for
      embedded software (2)}%
    {
      Suite du stage de Magistère de troisième année de Licence.\\
      Définition et implémentation d'une solution plus efficace pour
      le monitoring en ligne de propriétés temporelles sur processeur
      ARM. Expérimentations sur diverses études de cas.\\
      Technologies utilisées :
      \vspace{.4cm}
      \begin{cvitems}
        \item C/C++
        \item Awk/Sed
        \item GNU objdump/readelf
        \item DWARF debug informations
        \end{cvitems}
        \vspace{.4cm}
    }%
    {Mai 2018 - Août 2018} \cvpublication{Assertion-based Verification
      through Binary Instrumentation}%
    {E.Brignon \& L.Pierre}%
    {DATE'2019}%
    {Florence (Italie)}%
    {Mars 2019}
  \end{cvdescription}
}
}

\vspace{.5cm}

\cventrysep[3.5cm]{2016 - 2017}{%
  \vspace{-.5cm}
\cventry
{Licence d'informatique\\Première année de magistère informatique} % Degree
{Université Grenoble Alpes, UFR IM$^2$AG,Saint Martin d'Hères, France} % Institution
{} % Location
{} % Date(s)
{
  \begin{cvdescription}
    \cvinternship{Stage de Magistère, Laboratoire TIMA (Grenoble)}%
    {Solutions to implement safety/security online monitoring for
      embedded software (1)}%
    {Étude de la spécification formelle de propriétés temporelles pour
      le logiciel embarqué, définition et prototypage d'une première
      solution de monitoring en ligne pour de telles propriétés, sur
      processeur ARM.\\
    Technologies utilisées :
      \vspace{.4cm}
      \begin{cvitems}
        \item C/C++
        \item Awk/Sed
        \item GNU objdump/readelf
        \item DWARF debug informations
        \end{cvitems}
        \vspace{.4cm}}%
    {Juin 2017 - Août 2017}
  \end{cvdescription}
}
}

\vspace{.5cm}

\cventrysep[.5cm]{2014 - 2016}{
  \vspace{-.2cm}
\cventry
{Licence 1 et 2 Mathématique-Informatique internationale} % Degree
{Université Grenoble Alpes, DLST, Saint Martin d'Hères, France} % Institution
{} % Location
{} % Date(s)
{Formation mathématique et informatique en anglais.}
}

%------------------------------------------------

\end{cventries}
%  LocalWords:  Mathématique-Informatique
