%%%%%%%%%%%%%%%%%%%%%%%%%%%%%%%%%%%%%%%%%
% Awesome Resume/CV
% XeLaTeX Template
% Version 1.2 (27/3/2017)
%
% This template has been downloaded from:
% http://www.LaTeXTemplates.com
%
% Original author:
% Claud D. Park (posquit0.bj@gmail.com) with modifications by
% Vel (vel@latextemplates.com)
%
% License:
% CC BY-NC-SA 3.0 (http://creativecommons.org/licenses/by-nc-sa/3.0/)
%
% Important note:
% This template must be compiled with XeLaTeX, the below lines will ensure this
%!TEX TS-program = xelatex
%!TEX encoding = UTF-8 Unicode
%
%%%%%%%%%%%%%%%%%%%%%%%%%%%%%%%%%%%%%%%%%

%----------------------------------------------------------------------------------------
%	PACKAGES AND OTHER DOCUMENT CONFIGURATIONS
%----------------------------------------------------------------------------------------

\documentclass[11pt, a4paper]{awesome-cv} % A4 paper size by default, use 'letterpaper' for US letter
\usepackage{lipsum}
\usepackage{scrextend}
\geometry{left=2cm, top=1.5cm, right=2cm, bottom=2cm, footskip=.5cm} % Configure page margins with geometry

\fontdir[fonts/] % Specify the location of the included fonts

% Color for highlights
\colorlet{awesome}{awesome-darknight} % Default colors include: awesome-emerald, awesome-skyblue, awesome-red, awesome-pink, awesome-orange, awesome-nephritis, awesome-concrete, awesome-darknight
%\definecolor{awesome}{HTML}{CA63A8} % Uncomment if you would like to specify your own color

% Colors for text - uncomment and modify
%\definecolor{darktext}{HTML}{414141}
%\definecolor{text}{HTML}{414141}
%\definecolor{graytext}{HTML}{414141}
%\definecolor{lighttext}{HTML}{414141}

\renewcommand{\acvHeaderSocialSep}{\quad\textbar\quad} % If you would like to change the social information separator from a pipe (|) to something else

%----------------------------------------------------------------------------------------
%	PERSONAL INFORMATION
%	Comment any of the lines below if they are not required
%----------------------------------------------------------------------------------------

\name{Enzo}{Brignon}
\address{11 rue du bois taillis, 38610 Gi\`eres, France}
\mobile{(+33) 6 28 82 34 84}

\email{enzo.brignon@etu.univ-grenoble-alpes.fr}
% \homepage{www.posquit0.com}
\github{gueckmooh}
% \linkedin{posquit0}
%\skype{skypeid}
%\stackoverflow{SOid}{SOname}
%\twitter{@twit}
%\linkedin{linkedin name}
%\reddit{reddit account}
%\xing{xing name}
%\extrainfo{test} % Other text you want to include on this line

\position{} % Your expertise/fields
% \quote{``Make the change that you want to see in the world."} % A quote or statement

\makecvfooter{\today}{Enzo Brignon~~~·~~~CV}{\thepage} % Specify the letter footer with 3 arguments: (<left>, <center>, <right>), leave any of these blank if they are not needed

%----------------------------------------------------------------------------------------

\begin{document}

\makecvheader % Print the header

%----------------------------------------------------------------------------------------
%	CV/RESUME CONTENT
%	Each section is imported separately, open each file in turn to modify content
%----------------------------------------------------------------------------------------

%----------------------------------------------------------------------------------------
%	SECTION TITLE
%----------------------------------------------------------------------------------------

\cvsection{Education}

%----------------------------------------------------------------------------------------
%	SECTION CONTENT
%----------------------------------------------------------------------------------------

\begin{cventries}

%------------------------------------------------

\cventrysep{2018 - 2019}

\cventry
{Master 2 MOSIG (Master of Science in Informatics at Grenoble)} % Degree
{Université Grenoble Alpes, UFG IM$^2$AG} % Institution
{Saint Martin d'Hères, France} % Location
{Septembre 2018 - Juin 2019} % Date(s)
{}

\cventry {Troisième année de magistère informatique}%
{}%
{}%
{}%
{
  \begin{cvdescription}
    \cvinternship{Stage de Master 2 et Magistère}%
    {Vers l'analyse automatisée de tolérance aux fautes pour automates
      programmables}%
    {Étude sur la modélisation de scénarios de test nominaux et de
      scénarios de fautes, et génération automatique de tests
      exécutables.}%
    {Février 2019 - Août 2019}
  \end{cvdescription}
}


%------------------------------------------------

\end{cventries}
%----------------------------------------------------------------------------------------
%	SECTION TITLE
%----------------------------------------------------------------------------------------

\cvsection{Compétences}

%----------------------------------------------------------------------------------------
%	SECTION CONTENT
%----------------------------------------------------------------------------------------

\begin{cvskills}

%------------------------------------------------

\cvskill
{Programmation} % Category
{C/C++, Bash, Ocaml, Ada, Lua, Python, Awk, Assembleur ARM, JAVA} % Skills

%------------------------------------------------

\cvskill
{Méthodes formelles} % Category
{Logique temporelles, algèbres de processus, SAT/SMT} % Skills

% ------------------------------------------------

\cvskill
{Compilation}%
{Théorie des langages, analyse syntaxique et sémantique, édition de liens}

% --------------------------------------------------

\cvskill
{Calcul distribué}%
{OpenMP, MPI, JAVA RMI, RabbitMQ}

% --------------------------------------------------

\cvskill
{Bases de données}%
{SQL, MongoDB}

% --------------------------------------------------

\cvskill
{Outils}%
{Toolchain de compilation/debug pour ARM/x86, Git, Emacs, \LaTeX}

% --------------------------------------------------

\cvskill
{Langues} % Category
{Français, Anglais (lu, écrit, parlé)} % Skills

%------------------------------------------------

\end{cvskills}

\newgeometry{top=3cm, bottom=3cm, left=3cm, right=3cm}

\begin{addmargin}{-1cm}
  \cvsection{Motivations pour la thèse intitulée ``Analyse en ligne de propriétés temporelles
    sur SoPCpour plateformes satellitaires''}
\end{addmargin}

\vspace{1.4cm}

Madame, monsieur,\\

J'ai récemment terminé ma deuxième année de Master d'informatique à
l'Université Grenoble Alpes et je souhaite effectuer un doctorat en
contrat CIFRE avec Thales Alenia Space sur le sujet ``Analyse en ligne
de propriétés temporelles sur SoPCpour plateformes
satellitaires''. Après le bac j'ai effectué un L1--L2 Mathématiques
Informatique International puis un L3 Informatique. J'ai poursuivi par
une première année de Master Informatique et par un M2 MoSIG (Master
of Science in Informatics at Grenoble).

Mon choix pour un L1--L2 MIN international était motivé par le fait
que la pratique de l'anglais me paraissait importante si je voulais
faire de l'informatique dans le futur. En rentrant en licence je
savais déjà que je voulais travailler dans l'informatique mais je
n'avais pas encore décidé vers quelle branche me tourner. C'est
grâce aux enseignants chercheurs passionnés que je me suis intéressé
à la recherche.

Pour le L3, les choix de l'informatique ou des mathématiques
appliquées m'étaient offerts et ma décision de finir mon cursus
licence en informatique était naturel du fait que cette discipline
m'attirait le plus.

Pendant ma troisième année de licence je me suis inscrit en
Magistère pour découvrir ce qu'était la réalité du ``monde de la
recherche'' et savoir si cela pouvait m'intéresser vraiment.

Je suis autant intéressé par les aspects formels de l'informatique
que par les aspects plus pratiques tels que le système et les
interactions avec la machine. Laurence Pierre était la responsable
de l'UE Architectures Logicielles et Matérielles qui est une matière à
laquelle je porte beaucoup d'intérêt. C'est pour cette raison que
j'ai cherché mon stage de magistère de L3 auprès d'elle.

J'ai travaillé à TIMA sur un projet visant à automatiser la
vérification dynamique de propriétés temporelles sur des programmes
embarqués. Ma mission était de proposer un mécanisme d'observation
d'évènements dans l'exécution du programme, plus efficace que celui
déjà implémenté par l'équipe. J'ai pu définir un mécanisme
d'observation utilisant des interruptions logicielles, faire des
expérimentations sur plusieurs cartes et spécifier les modifications
à apporter à l'outil d'instrumentation OSIRIS, de manière à ce qu'il
puisse mettre en place la nouvelle solution.

% Ce stage m'a permis de mieux découvrir le monde de la recherche et
% la rigueur de raisonnement nécessaire. Cela m'a beaucoup plu,
% j'étais enfin sûr de vouloir continuer dans cette voie.

L'année suivante je suis entré en première année de master et j'ai
continué le magistère en plus du stage de M1 sur la suite du projet
sur lequel j'ai travaillé en L3. Le mécanisme utilisant les
interruptions logicielles étant peu approprié en présence d'un système
d'exploitation, nous avons donc étendu la méthode avec une solution
alternative basée sur la moditifaction des fichiers relogeables pour y
insérer des instructions de branchements permettant d'appeler des
fonctions d'observation. Après avoir spécifié les modifications à
apporter à OSIRIS, je l'ai modifié pour qu'il puisse mettre en place
les différents mécanismes d'observation, et ai fait des tests
comparatifs. Ces travaux ont aboutis à la publication d'un article à
la conférence DATE'2019. La nouvelle méthode implémentée donne de très
bons résultats, réduisant grandement le surcoût en temps du mécanisme
d'observation. Cependant une telle observation n'a été définie que
pour des programmes non optimisés et pourrait être portée sur un plus
large spectre d'applications (exécutables optimisés, autres
architectures).

Cette nouvelle expérience m'a fait prendre conscience de l'importance
de la modélisation et la vérification de programmes et systèmes et que
ce domaine m'attire beaucoup.

L'année suivante j'ai donc fait le M2 MoSIG spécialité High confidence
Embedded and Cyber-physical Systems et le programme de ce parcours m'a
permis d'approfondir mes connaissances dans ce domaine.

J'ai effectué mon stage de fin de Master, en plus de ma troisième
année de Magistère avec Laurence Pierre sur le sujet de la génération
automatique de tests et l'analyse de tolérance aux fautes pour des
automates programmables, sujet qui découlait d'une opportunité de
collaboration avec une entreprise locale. Cette expérience m'a permis
d'acquérir des connaissances supplémentaires dans le domaine des
méthodes formelles notamment sur le model checking et l'analyse de
tolérance aux fautes.

% L'année prochaine, je souhaite faire un
% Master 2 MOSIG en spécialité High confidence Embedded and
% Cyber-physical Systems car le programme de ce parcours me permettra
% d'approfondir mes connaissances dans ce domaine. Il est prévu que je
% continue en stage de fin de Master et en Magistère à TIMA avec
% Laurence Pierre dans la continuité des précédents travaux.

% Mon expérience à TIMA ainsi que les options que j'ai choisies
% pendant ma première année de master m'ont permis d'améliorer mes
% compétences en conception logicielle et matérielle notamment grâce à
% l'option Introduction à la modélisation et à la vérification des
% systèmes numériques donnée par Laurence Pierre. J'ai également
% apprécié la composante plus formelle de l'informatique avec les
% cours de complexité et de calculabilité pendant lesquels j'ai
% compris l'importance d'avoir une base solide dans la théorie qui
% fonde l'informatique.

% La vérification des systèmes cyber-physiques est un domaine qui
% entremêle formalisme et interactions avec la machine et le monde
% physique, deux sujets qui m'intéressent énormément. Je souhaite donc
% poursuivre en thèse après mon Master sous la supervision de Laurence
% Pierre dans ce domaine qui me passionne. C'est pour cela que je fais
% la demande de bourse de Master~2 Persyval-Lab.

La vérification des systèmes cyber-physiques est un domaine qui
entremêle formalisme et interactions avec la machine et le monde
physique, deux sujets qui m'intéressent énormément. Le sujet proposé
pour une thèse CIFRE avec Thales Alenia Space sur l'``Analyse en ligne
de propriétés temporelles sur SoPCpour plateformes satellitaires'' est
en adéquation avec mes attentes et mes compétences et me permettrai de
continuer de travailler sur le sujet de l'``Assertion Based
Verification'' sur lequel j'ai travaillé pendant mes stages de L3 et
M1. Travailler en collaboration avec Thales Alenia Space est


Merci pour votre temps et votre considération.


% %----------------------------------------------------------------------------------------
%	SECTION TITLE
%----------------------------------------------------------------------------------------

\cvsection{Expérience Professionnelle}

%----------------------------------------------------------------------------------------
%	SECTION CONTENT
%----------------------------------------------------------------------------------------

\begin{cventries}

%------------------------------------------------

\cventry
{Animateur de séjour adapté} % Job title
{ALPAS (Association de Loisirs et de Promotion des Activités Sociales)} % Organization
{Grenoble} % Location
{Étés 2015, 2016, 2017} % Date(s)
{ % Description(s) of tasks/responsibilities
\begin{cvitems}
Gestion d'un lieu de vie (ménage, cuisine, lessive, etc...).
Animation d'activités avec des adultes déficients intellectuels.
\end{cvitems}
}

% --------------------------------------------------

\end{cventries}

% \newpage % Force a new page for looks

% \input{cv-sections/extracurricular.tex}
% \input{cv-sections/honors.tex}
% \input{cv-sections/presentation.tex}
% \input{cv-sections/writing.tex}
% \input{cv-sections/committees.tex}

% ----------------------------------------------------------------------------------------

\end{document}